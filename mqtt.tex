\documentclass[xcolor={x11names}]{beamer}
\usetheme{Madrid}

\usepackage{amssymb}
\usepackage{ulem}
\usepackage[utf8]{inputenc}
\usepackage{mathtools}
\usepackage{multicol}
%\usepackage[x11names]{xcolor}
\usefonttheme{professionalfonts}

% Code
\definecolor{codegreen}{rgb}{0,0.6,0}
\definecolor{codegray}{rgb}{0.5,0.5,0.5}
\definecolor{codepurple}{rgb}{0.58,0,0.82}
\definecolor{backcolour}{rgb}{0.95,0.95,0.92}
\usepackage{listings}
\lstdefinestyle{mystyle}{
    backgroundcolor=\color{backcolour},
    commentstyle=\color{codegreen},
    keywordstyle=\color{magenta},
    numberstyle=\tiny\color{codegray},
    stringstyle=\color{codepurple},
    basicstyle=\ttfamily\footnotesize,
    breakatwhitespace=false,
    breaklines=true,
    captionpos=b,
    keepspaces=true,
    numbers=left,
    numbersep=5pt,
    showspaces=false,
    showstringspaces=false,
    showtabs=false,
    tabsize=2
}
\lstset{style=mystyle}


% Subfigures
\usepackage{caption}
\usepackage{subcaption}


% License
\usepackage[
    type={CC},
    modifier={by-nc-sa},
    version={4.0},
    imagewidth=4pt
]{doclicense}




% Change base colour beamer@blendedblue (originally RGB: 0.2,0.2,0.7)
% \colorlet{beamer@blendedblue}{DarkSeaGreen4}







%% MATH commands
\DeclareMathOperator{\Var}{Var}


%% THEOREMS
%\newtheorem{theorem}{Theorem}
\newtheorem{thm}{Teorema}[section] % the main one
% Definición
%\theoremstyle{definition}
\newtheorem{definicion}{Definición}[section]
\newtheorem{lema}{Lema}[section]



%% PFGplots %%
\usepackage{pgfplots}

%% Exponential distribution
\pgfmathdeclarefunction{exponential}{1}{%
  \pgfmathparse{(#1)*exp(-#1*x)}%
}
\pgfmathdeclarefunction{exponentialcdf}{1}{%
  \pgfmathparse{1-exp(-#1*x)}%
}

%% Poisson distribution
\pgfmathdeclarefunction{poiss}{1}{%
  \pgfmathparse{(#1^x)*exp(-#1)/(x!)}%
}

%% Normal distribution (#1=mu, #2=sigma)
% John D. Cook approx. https://tex.stackexchange.com/a/124629
\pgfmathdeclarefunction{normalcdf}{2}{%
  \pgfmathparse{1/(1 + exp(-0.07056*((x-#1)/#2)^3 - 1.5976*(x-#1)/#2))}%
}




\newcommand{\red}[1]{{\color{red}#1}}
\newcommand{\blue}[1]{{\color{blue}#1}}

%%%%%%%%%%
%% TIKZ %%
%%%%%%%%%%
\usepackage{tikz}
\usepackage{animate}
\usetikzlibrary{positioning}
\usetikzlibrary{shapes,arrows, positioning, calc}
\usetikzlibrary{overlay-beamer-styles}
\usetikzlibrary{chains,shapes.multipart}
\usetikzlibrary{scopes}
\usetikzlibrary{automata}
\usetikzlibrary{positioning}  %                 ...positioning nodes
\usetikzlibrary{arrows}       %                 ...customizing arrows
\usetikzlibrary{intersections}


%%%%%%%%%
%% PGF %%
%%%%%%%%%
\usepgfplotslibrary{fillbetween}


%%% Insert section name before the section %%%
\AtBeginSection[]{
  \begin{frame}
  \vfill
  \centering
  \begin{beamercolorbox}[sep=8pt,center,shadow=true,rounded=true]{title}
    \usebeamerfont{title}\insertsectionhead\par%
  \end{beamercolorbox}
  \vfill
  \end{frame}
}



\title[MQTT]{MQTT: fundamentos del protocolo}
%% \subtitle{Redes y Servicios de Telecomunicaciones (RSTC)\\
%% Grado en Ingeniería de Tecnologías y Servicios de Telecomunicación}
\titlegraphic{%
\doclicenseIcon {\tiny \hspace{1em}\doclicenseText}
}

\author{\textcolor{white}{RSER curso 2024-2025}}
%\author{Jorge Martín Pérez\inst{1}}
%\institute{
%    \inst{1}
%    Departamento de Ingeniería Telemática, Universidad Politécnica de Madrid
%}

\date{\today}







%%%%%%%%%%%%%%%%%%%%
%%% SLIDES START %%%
%%%%%%%%%%%%%%%%%%%%
\begin{document}


%%% TITLE %%%
\frame{\titlepage }


\begin{frame}[allowframebreaks]{Contenido}
    \tableofcontents
\end{frame}




\section{Introducción}
\begin{frame}{\secname}
    MQTT (Message Queuing Telemetry Transport)
    se utiliza para:
    \begin{itemize}
        \item transportar mensajes (e.g. pulsaciones);
        \item utilizando esquema \textbf{publish/subscribe}.
    \end{itemize}

    \vfill
    Vamos a ver:
    \begin{itemize}
        \item en qué consiste el \textbf{publish/subscribe}; y
        \item cómo funciona el protocolo.
    \end{itemize}
\end{frame}




\begin{frame}{\secname}
    Entidades modelo publish/subscribe:
    \begin{itemize}
        \item \textbf{Publisher}: publica mensajes,
            e.g. 60\,\textrm{BPM};
        \item \textbf{Broker}: recibe, almacena, y envía mensajes;
        \item \textbf{Subscriber}: recibe mensajes.
    \end{itemize}
    \vspace{1em}
    \begin{figure}
        \begin{tikzpicture}
    \node[draw,circle,fill=blue!20] (pub1) at (0,0) {PUB};

    \node[draw,fill=yellow!20] (broker)
        at ($(pub1)+(3,0)$) {Broker};

    \node[draw,circle,fill=red!20] (subs)
        at ($(broker)+(3,0)$) {SUB};


    \draw[->] (pub1.east) --
        node[midway,above] {publish}
        (broker.west);
    \draw[->] (broker.east) --
        node[midway,above] {subscribe}
        (subs.west);
\end{tikzpicture}

    \end{figure}
\end{frame}



\begin{frame}{\secname}
    En qué casos se usa publish/subscribe:
    \begin{itemize}
        \item comunicaciones tolerantes a \textbf{latencias};
        \item escenarios con \textbf{muchos dispositivos}
            (escalabilidad).
    \end{itemize}
    \vspace{1em}
    \begin{figure}
        \begin{tikzpicture}
    \node[draw,circle,fill=blue!20] (pub1) at (0,0) {PUB};

    \node[draw,fill=yellow!20] (broker)
        at ($(pub1)+(3,0)$) {Broker};

    \node[draw,circle,fill=red!20] (subs)
        at ($(broker)+(3,0)$) {SUB};


    \draw[->] (pub1.east) --
        node[midway,above] {publish}
        (broker.west);
    \draw[->] (broker.east) --
        node[midway,above] {subscribe}
        (subs.west);
\end{tikzpicture}

    \end{figure}
\end{frame}




\section{Topics}
\begin{frame}{\secname}
    Los \texttt{topics} son como ``buzones''
    en los que se publican mensajes:
    \begin{itemize}
        \item tienen \textbf{jerarquía}.
    \end{itemize}

    \vspace{1em}

    \begin{minipage}{.5\textwidth} %
        {\color{Firebrick3}Madrid}/{\color{Gold3}Comillas}/{\color{DodgerBlue3}Laura}/{\color{OliveDrab4}Presión}
        {\color{Firebrick3}Madrid}/{\color{Gold3}Comillas}/{\color{DodgerBlue3}Laura}/{\color{OliveDrab4}Pulso}
        {\color{Firebrick3}Madrid}/{\color{Gold3}Comillas}/{\color{DodgerBlue3}Sofía}/{\color{OliveDrab4}Pulso}
        {\color{Firebrick3}Madrid}/{\color{Gold3}Alcorcón}/{\color{DodgerBlue3}Sofía}/{\color{OliveDrab4}Pulso}
    \end{minipage}
    \begin{minipage}{.45\textwidth} %
        \begin{figure}
            \begin{tikzpicture}
    \node[draw,fill=OliveDrab3!20]
        (prele) at (0,0) {Presión};
    \node[draw,fill=OliveDrab3!20,
        anchor=west]
        (pulsole) at
        ($(prele.east)+(.1,0)$)
        {Pulso};
    \node[draw,fill=OliveDrab3!20,
        anchor=west]
        (preri) at
        ($(pulsole.east)+(.1,0)$)
        {Presión};
    \node[draw,fill=OliveDrab3!20,
        anchor=west]
        (pulsori) at
        ($(preri.east)+(.1,0)$)
        {Pulso};

    \node[draw,fill=DodgerBlue3!20,
        anchor=south]
        (laura)
        at
        ($(prele)+(.5,.75)$)
        {Laura};
    \node[draw,fill=DodgerBlue3!20,
        anchor=south]
        (sofia)
        at
        ($(preri)+(.5,.75)$)
        {Sofía};



    \node[draw,fill=Gold2!20,
        anchor=south]
        (comillas)
        at
        ($(laura)+(.5,.75)$)
        {Comillas};
    \node[draw,fill=Gold2!20,
        anchor=west]
        (alcorcon)
        at
        ($(comillas.east)+(.5,0)$)
        {Alcorcón};

    \node[draw,fill=Firebrick3!20,
        anchor=south]
        (madrid)
        at
        ($(comillas)+(.5,.75)$)
        {Madrid};


    \draw (madrid.south) -- (comillas.north);
    \draw (madrid.south) -- (alcorcon.north);
    \draw (comillas.south) -- (laura.north);
    \draw (comillas.south) -- (sofia.north);
    \draw (alcorcon.south) -- (sofia.north);
    \draw (laura.south) -- (prele.north);
    \draw (laura.south) -- (pulsole.north);
    \draw (sofia.south) -- (preri.north);
    \draw (sofia.south) -- (pulsori.north);

        

\end{tikzpicture}

        \end{figure}
    \end{minipage}
\end{frame}




\begin{frame}{\secname}
    \textbf{Single level wildcard (+)}:
    nos permite sustituir un solo nivel.

    \vspace{1em}

    
    \begin{minipage}{.51\textwidth} %
        {\color{Firebrick3}Madrid}/{\color{Gold3}Alcorcón}/{\color{DodgerBlue3}\textbf{+}}/{\color{OliveDrab4}Pulso} resulta en:

        \begin{itemize}
            \item {\color{Firebrick3}Madrid}/{\color{Gold3}Alcorcón}/{\color{DodgerBlue3}\textbf{Sofía}}/{\color{OliveDrab4}Pulso}
        \end{itemize}
    \end{minipage}
    \begin{minipage}{.4\textwidth} %
        \begin{figure}
            \begin{tikzpicture}
    \node[draw,fill=OliveDrab3!20]
        (prele) at (0,0) {Presión};
    \node[draw,fill=OliveDrab3!20,
        anchor=west]
        (pulsole) at
        ($(prele.east)+(.1,0)$)
        {Pulso};
    \node[draw,fill=OliveDrab3!20,
        anchor=west]
        (preri) at
        ($(pulsole.east)+(.1,0)$)
        {Presión};
    \node[draw,fill=OliveDrab3!20,
        anchor=west]
        (pulsori) at
        ($(preri.east)+(.1,0)$)
        {Pulso};

    \node[draw,fill=DodgerBlue3!20,
        anchor=south]
        (laura)
        at
        ($(prele)+(.5,.75)$)
        {Laura};
    \node[draw,fill=DodgerBlue3!20,
        anchor=south]
        (sofia)
        at
        ($(preri)+(.5,.75)$)
        {Sofía};



    \node[draw,fill=Gold2!20,
        anchor=south]
        (comillas)
        at
        ($(laura)+(.5,.75)$)
        {Comillas};
    \node[draw,fill=Gold2!20,
        anchor=west]
        (alcorcon)
        at
        ($(comillas.east)+(.5,0)$)
        {Alcorcón};

    \node[draw,fill=Firebrick3!20,
        anchor=south]
        (madrid)
        at
        ($(comillas)+(.5,.75)$)
        {Madrid};


    \draw (madrid.south) -- (comillas.north);
    \draw (madrid.south) -- (alcorcon.north);
    \draw (comillas.south) -- (laura.north);
    \draw (comillas.south) -- (sofia.north);
    \draw (alcorcon.south) -- (sofia.north);
    \draw (laura.south) -- (prele.north);
    \draw (laura.south) -- (pulsole.north);
    \draw (sofia.south) -- (preri.north);
    \draw (sofia.south) -- (pulsori.north);

        
    \draw[ultra thick] (madrid.south)
        to[out=-10,in=90]
        (alcorcon.north)
        to[out=-90,in=90]
        (pulsori.north);

\end{tikzpicture}

        \end{figure}
    \end{minipage}
\end{frame}








\begin{frame}{\secname}
    \textbf{Multi level wildcard (\#)}:
    nos permite sustituir muchos niveles.

    \vspace{1em}

    
    \begin{minipage}{.51\textwidth} %
        {\color{Firebrick3}Madrid}/{\color{Gold3}Alcorcón}/{\color{DodgerBlue3}\textbf{\#}} resulta en:

        \begin{itemize}
            \item {\color{Firebrick3}Madrid}/{\color{Gold3}Alcorcón}/{\color{DodgerBlue3}\textbf{Sofía}}/{\color{OliveDrab4}Pulso}
            \item {\color{Firebrick3}Madrid}/{\color{Gold3}Alcorcón}/{\color{DodgerBlue3}\textbf{Sofía}}/{\color{OliveDrab4}Presión}
        \end{itemize}
    \end{minipage}
    \begin{minipage}{.4\textwidth} %
        \begin{figure}
            \begin{tikzpicture}
    \node[draw,fill=OliveDrab3!20]
        (prele) at (0,0) {Presión};
    \node[draw,fill=OliveDrab3!20,
        anchor=west]
        (pulsole) at
        ($(prele.east)+(.1,0)$)
        {Pulso};
    \node[draw,fill=OliveDrab3!20,
        anchor=west]
        (preri) at
        ($(pulsole.east)+(.1,0)$)
        {Presión};
    \node[draw,fill=OliveDrab3!20,
        anchor=west]
        (pulsori) at
        ($(preri.east)+(.1,0)$)
        {Pulso};

    \node[draw,fill=DodgerBlue3!20,
        anchor=south]
        (laura)
        at
        ($(prele)+(.5,.75)$)
        {Laura};
    \node[draw,fill=DodgerBlue3!20,
        anchor=south]
        (sofia)
        at
        ($(preri)+(.5,.75)$)
        {Sofía};



    \node[draw,fill=Gold2!20,
        anchor=south]
        (comillas)
        at
        ($(laura)+(.5,.75)$)
        {Comillas};
    \node[draw,fill=Gold2!20,
        anchor=west]
        (alcorcon)
        at
        ($(comillas.east)+(.5,0)$)
        {Alcorcón};

    \node[draw,fill=Firebrick3!20,
        anchor=south]
        (madrid)
        at
        ($(comillas)+(.5,.75)$)
        {Madrid};


    \draw (madrid.south) -- (comillas.north);
    \draw (madrid.south) -- (alcorcon.north);
    \draw (comillas.south) -- (laura.north);
    \draw (comillas.south) -- (sofia.north);
    \draw (alcorcon.south) -- (sofia.north);
    \draw (laura.south) -- (prele.north);
    \draw (laura.south) -- (pulsole.north);
    \draw (sofia.south) -- (preri.north);
    \draw (sofia.south) -- (pulsori.north);

        
    \draw[ultra thick] (madrid.south)
        to[out=-10,in=90]
        (alcorcon.north)
        to[out=-90,in=90]
        (pulsori.north);
    \draw[ultra thick] (madrid.south)
        to[out=-25,in=70]
        (preri.north);

\end{tikzpicture}

        \end{figure}
    \end{minipage}
\end{frame}






\section{PUBLISH}
\begin{frame}{\secname}
    El publisher envía información mediante
    mensajes de tipo \texttt{PUBLISH}.

    \vspace{1em}

    \begin{figure}
        \begin{tikzpicture}
    \node[draw,circle,fill=DodgerBlue3!20] (pub1) at (0,0) {PUB};

    \node[draw,fill=yellow!20] (broker)
        at ($(pub1)+(9,0)$) {Broker};


    \draw[->] (pub1.east) --
        node[midway,above,draw,align=center]
        (pubmsg)
        {PUBLISH\\
        Topic:{\color{Firebrick3}Madrid}/{\color{Gold3}Comillas}/{\color{DodgerBlue3}Laura}/{\color{OliveDrab4}Pulso}\\
        Message: 60\,\texttt{BMP}}
        (broker.west);

    \node[anchor=north] at
        ($(broker.south)-(0,.5)$)
    {
        \scalebox{.5}{
            \begin{tikzpicture}
    \node[draw,fill=OliveDrab3!20]
        (prele) at (0,0) {Presión};
    \node[draw,fill=OliveDrab3!20,
        anchor=west]
        (pulsole) at
        ($(prele.east)+(.1,0)$)
        {Pulso};
    \node[draw,fill=OliveDrab3!20,
        anchor=west]
        (preri) at
        ($(pulsole.east)+(.1,0)$)
        {Presión};
    \node[draw,fill=OliveDrab3!20,
        anchor=west]
        (pulsori) at
        ($(preri.east)+(.1,0)$)
        {Pulso};

    \node[draw,fill=DodgerBlue3!20,
        anchor=south]
        (laura)
        at
        ($(prele)+(.5,.75)$)
        {Laura};
    \node[draw,fill=DodgerBlue3!20,
        anchor=south]
        (sofia)
        at
        ($(preri)+(.5,.75)$)
        {Sofía};



    \node[draw,fill=Gold2!20,
        anchor=south]
        (comillas)
        at
        ($(laura)+(.5,.75)$)
        {Comillas};
    \node[draw,fill=Gold2!20,
        anchor=west]
        (alcorcon)
        at
        ($(comillas.east)+(.5,0)$)
        {Alcorcón};

    \node[draw,fill=Firebrick3!20,
        anchor=south]
        (madrid)
        at
        ($(comillas)+(.5,.75)$)
        {Madrid};


    \draw (madrid.south) -- (comillas.north);
    \draw (madrid.south) -- (alcorcon.north);
    \draw (comillas.south) -- (laura.north);
    \draw (comillas.south) -- (sofia.north);
    \draw (alcorcon.south) -- (sofia.north);
    \draw (laura.south) -- (prele.north);
    \draw (laura.south) -- (pulsole.north);
    \draw (sofia.south) -- (preri.north);
    \draw (sofia.south) -- (pulsori.north);

        

\end{tikzpicture}

        }
    };

    \draw
        (broker.south)
        to[out=-130,in=90]
        ($(broker)-(2,1)$)
        to[out=-90,in=-180]
        ($(broker)-(.7,3)$)
        node[anchor=west,draw,inner sep=1]
        (msgstore)
        {\tiny Message}
        ;

    \draw[->]
        (msgstore) -- ($(msgstore)+(0,.3)$);
\end{tikzpicture}

    \end{figure}
    
\end{frame}



\begin{frame}{\secname}
    El broker anuncia información
    a los suscriptores con
    mensajes \texttt{PUBLISH}.

    \vspace{1em}

    \begin{figure}
        \begin{tikzpicture}
    \node[draw,fill=yellow!20]
        (broker) at (0,0) {Broker};

    \node[draw,circle,fill=Firebrick3!20]
        (subs)
        at ($(broker)+(9,0)$) {SUB};



    \draw[->] (broker.east) --
        node[midway,above,draw,align=center]
        (pubmsg)
        {PUBLISH\\
        Topic:{\color{Firebrick3}Madrid}/{\color{Gold3}Comillas}/{\color{DodgerBlue3}Laura}/{\color{OliveDrab4}Pulso}\\
        Message: 60\,\texttt{BMP}}
        (subs.west);

    \node[anchor=north] at
        ($(broker.south)-(0,.5)$)
    {
        \scalebox{.5}{
            \begin{tikzpicture}
    \node[draw,fill=OliveDrab3!20]
        (prele) at (0,0) {Presión};
    \node[draw,fill=OliveDrab3!20,
        anchor=west]
        (pulsole) at
        ($(prele.east)+(.1,0)$)
        {Pulso};
    \node[draw,fill=OliveDrab3!20,
        anchor=west]
        (preri) at
        ($(pulsole.east)+(.1,0)$)
        {Presión};
    \node[draw,fill=OliveDrab3!20,
        anchor=west]
        (pulsori) at
        ($(preri.east)+(.1,0)$)
        {Pulso};

    \node[draw,fill=DodgerBlue3!20,
        anchor=south]
        (laura)
        at
        ($(prele)+(.5,.75)$)
        {Laura};
    \node[draw,fill=DodgerBlue3!20,
        anchor=south]
        (sofia)
        at
        ($(preri)+(.5,.75)$)
        {Sofía};



    \node[draw,fill=Gold2!20,
        anchor=south]
        (comillas)
        at
        ($(laura)+(.5,.75)$)
        {Comillas};
    \node[draw,fill=Gold2!20,
        anchor=west]
        (alcorcon)
        at
        ($(comillas.east)+(.5,0)$)
        {Alcorcón};

    \node[draw,fill=Firebrick3!20,
        anchor=south]
        (madrid)
        at
        ($(comillas)+(.5,.75)$)
        {Madrid};


    \draw (madrid.south) -- (comillas.north);
    \draw (madrid.south) -- (alcorcon.north);
    \draw (comillas.south) -- (laura.north);
    \draw (comillas.south) -- (sofia.north);
    \draw (alcorcon.south) -- (sofia.north);
    \draw (laura.south) -- (prele.north);
    \draw (laura.south) -- (pulsole.north);
    \draw (sofia.south) -- (preri.north);
    \draw (sofia.south) -- (pulsori.north);

        

\end{tikzpicture}

        }
    };

    \draw
        (broker.south)
        to[out=-130,in=90]
        ($(broker)-(2,1)$)
        to[out=-90,in=-180]
        ($(broker)-(.7,3)$)
        node[anchor=west,draw,inner sep=1]
        (msgstore)
        {\tiny Message}
        ;

    \draw[->]
        (msgstore) -- ($(msgstore)+(0,.3)$);
\end{tikzpicture}

    \end{figure}
\end{frame}



\section{SUBSCRIBE}
\begin{frame}{\secname}
    Un dispositivo se suscribe a
    un \texttt{topic} con
    mensajes \texttt{SUBSCRIBE}.

    \vspace{1em}

    \begin{figure}
        \begin{tikzpicture}
    \node[draw,fill=yellow!20]
        (broker) at (0,0) {Broker};

    \node[draw,circle,fill=Firebrick3!20]
        (subs)
        at ($(broker)+(9,0)$) {SUB};



    \draw[->] (subs.west) --
        node[midway,above,draw,align=center]
        (pubmsg)
        {SUBSCRIBE\\
        Topics:\\
        {\color{Firebrick3}Madrid}/{\color{Gold3}Comillas}/{\color{DodgerBlue3}Laura}/{\color{OliveDrab4}Pulso}\\
        {\color{Firebrick3}Madrid}/{\color{Gold3}Comillas}/{\color{DodgerBlue3}Laura}/{\color{OliveDrab4}Presión}}
        (broker.east);

\end{tikzpicture}

    \end{figure}


    \emph{Nota}: podríamos usar el wildcard
    {\color{Firebrick3}Madrid}/{\color{Gold3}Alcorcón}/{\color{DodgerBlue3}Laura}/{\color{OliveDrab4}\textbf{+}}
\end{frame}




\begin{frame}{\secname}
    En cuanto el broker recibe un
    mensaje nuevo, lo publica a los
    suscriptores.
    \begin{figure}
        \begin{tikzpicture}
    \foreach \i in {0,...,4}
        \node (pub\i) at (0,\i*.6) {};

    \node[draw,fill=yellow!20] (broker)
        at ($(pub2)+(3,0)$) {Broker};

    \foreach \i in {0,...,4}
        \node[draw,circle,fill=Firebrick3!20]
            (subs\i)
            at ($(pub\i)+(6,0)$) {SUB\i};


    \foreach \i in {0,...,4}{
        \draw[->] (broker.east) --
            node[above,midway] (suba\i) {}
            (subs\i.west);
    }

    \node[rotate=40] at (suba4.north)
        {\texttt{PUBLISH}};
\end{tikzpicture}

    \end{figure}
\end{frame}




\section{KEEPALIVE}
\begin{frame}{\secname}
    Los sensores sufren pérdidas de
    conexión. Las detectamos con
    \texttt{KEEPALIVE}.
    
    \begin{figure}
        \begin{tikzpicture}
    \node[draw,fill=yellow!20]
        (broker) at (0,0) {Broker};

    \node[draw,circle,fill=Firebrick3!20]
        (subs)
        at ($(broker)+(9,0)$) {SUB};



    \draw[->] (subs.west) --
        node[midway,above,draw,align=center]
        (pubmsg)
        {SUBSCRIBE\\
        Topics:\\
        {\color{Firebrick3}Madrid}/{\color{Gold3}Comillas}/{\color{DodgerBlue3}Laura}/{\color{OliveDrab4}Pulso}\\
        {\color{Firebrick3}Madrid}/{\color{Gold3}Comillas}/{\color{DodgerBlue3}Laura}/{\color{OliveDrab4}Presión}}
        (broker.east);



\end{tikzpicture}

    \end{figure}
\end{frame}


\begin{frame}{\secname}
    La conexión puede perderse del
    lado del suscriptor.
    \begin{figure}
        \begin{tikzpicture}
    \node[draw,fill=yellow!20]
        (broker) at (0,0) {Broker};
    \node[draw,circle,fill=Firebrick3!20]
        (subs)
        at ($(broker)+(9,0)$) {SUB};


    \draw[->] 
        (subs.west)
        --
        node[midway,above,Firebrick4]
        {¿Estás viva?}
        ($(broker.east)+(4,0)$)
        node[pos=1,Firebrick4,anchor=east]
        {\textbf{X}};


\end{tikzpicture}

    \end{figure}

    La conexión puede perderse del
    lado del broker.
    \begin{figure}
        \begin{tikzpicture}
    \node[draw,fill=yellow!20]
        (broker) at (0,0) {Broker};
    \node[draw,circle,fill=Firebrick3!20]
        (subs)
        at ($(broker)+(9,0)$) {SUB};

    \node[draw,fill=yellow!20]
        (broker2) at ($(broker)-(0,1.25)$) {Broker};
    \node[draw,circle,fill=Firebrick3!20]
        (subs2)
        at ($(broker2)+(9,0)$) {SUB};

    \draw[->] 
        (subs.west)
        --
        node[midway,above,Firebrick4]
        {¿Estás viva?}
        (broker.east);
    \draw[->] 
        (broker2.east)
        --
        node[midway,above,Gold4]
        {Sí}
        ($(subs2.west)-(4,0)$)
        node[pos=1,anchor=west,Firebrick4]
        {\textbf{X}}
        ;



\end{tikzpicture}

    \end{figure}
\end{frame}



\section{Formato Cabecera \& Flujos}
\begin{frame}{\secname}
    Campos de la cabecera MQTT:
    \begin{itemize}
        \item Packet Type: si es un PUBLISH, SUBSCRIBE, etc.;
        \item Flag: si el mensaje es duplicado y su QoS;
        \item Variable Header: cabeceras extra; y
        \item Variable Payload: el mensaje(s) como e.g. 
            60\,\textrm{BPM};
    \end{itemize}
    \begin{figure}
        \begin{tikzpicture}

\end{tikzpicture}

    \end{figure}
\end{frame}


\begin{frame}{\secname}
    Establecimiento de \textbf{conexión}:
    \begin{figure}
        \begin{tikzpicture}
    \node[draw,circle,fill=DodgerBlue3!20] (pub1) at (0,0) {PUB};

    \node[draw,fill=yellow!20] (broker)
        at ($(pub1)+(5,0)$) {Broker};

    \draw
        (pub1)
        --
        ($(pub1.south)-(0,2.3)$)
        ;
    \draw
        (broker)
        --
        ($(broker.south)-(0,2.5)$)
        ;


    \draw[->]
        ($(pub1)-(0,1)$)
        --
        node[midway,above] {\texttt{CONNECT}}
        ($(broker)-(0,1)$)
        ;
    \draw[->]
        ($(broker)-(0,2)$)
        --
        node[midway,above] {\texttt{CONNACK}}
        ($(pub1)-(0,2)$)
        ;
\end{tikzpicture}

    \end{figure}
    El mensaje \texttt{CONNECT} contiene
    información sobre:
    \begin{itemize}
        \item \texttt{clientId}: identificador de cliente;
        \item \texttt{keepAlive}: cada cuánto se comprueba
            conexión; y
        \item \texttt{lastQoS}: QoS deseada.
    \end{itemize}
\end{frame}





\begin{frame}{\secname}
    PINGs para comprobar si está vivo (el Keepalive).
    \begin{figure}
        \begin{tikzpicture}
    \node[draw,circle,fill=DodgerBlue3!20] (pub1) at (0,0) {PUB};

    \node[draw,fill=yellow!20] (broker)
        at ($(pub1)+(5,0)$) {Broker};

    \draw
        (pub1)
        --
        ($(pub1.south)-(0,2.3)$)
        ;
    \draw
        (broker)
        --
        ($(broker.south)-(0,2.5)$)
        ;


    \draw[->]
        ($(pub1)-(0,1)$)
        --
        node[midway,above] {\texttt{PINGREQ}}
        ($(broker)-(0,1)$)
        ;
    \draw[->]
        ($(broker)-(0,2)$)
        --
        node[midway,above] {\texttt{PINGREP}}
        ($(pub1)-(0,2)$)
        ;
\end{tikzpicture}

    \end{figure}
    Se intercambian:
    \begin{itemize}
        \item \texttt{PINGREQ}: pregunta para comprobar conexión;
        \item \texttt{PINGREP}: respuesta confirmando recepción.
    \end{itemize}
\end{frame}






\begin{frame}{\secname}
    \textbf{Suscripción} a un \texttt{topic}:
    \begin{figure}
        \begin{tikzpicture}
    \node[draw,fill=yellow!20] (broker)
        at (0,0) {Broker};

    \node[draw,circle,fill=Firebrick3!20]
        (sub1) at ($(broker)+(5,0)$) {SUB};

    \draw
        (sub1)
        --
        ($(sub1.south)-(0,1.4)$)
        ;
    \draw
        (broker)
        --
        ($(broker.south)-(0,1.7)$)
        ;


    \draw[->]
        ($(sub1)-(0,.7)$)
        --
        node[midway,above]
        {\texttt{SUBSCRIBE}}
        ($(broker)-(0,.7)$)
        ;
    \draw[->]
        ($(broker)-(0,1.4)$)
        --
        node[midway,above] {\texttt{SUBACK}}
        ($(sub1)-(0,1.4)$)
        ;
\end{tikzpicture}

    \end{figure}
    Como hemos visto, el \texttt{SUBSCRIBE}
    indica uno/varios \texttt{topics}:
    \begin{itemize}
        \item {\color{Firebrick3}Madrid}/{\color{Gold3}Comillas}/{\color{DodgerBlue3}\#}
        \item {\color{Firebrick3}Madrid}/{\color{Gold3}Comillas}/{\color{DodgerBlue3}Laura}/{\color{OliveDrab4}Pulso}
    \end{itemize}
\end{frame}



\begin{frame}{\secname}
    \textbf{Eliminar suscripción} a un \texttt{topic}:
    \begin{figure}
        \begin{tikzpicture}
    \node[draw,fill=yellow!20] (broker)
        at (0,0) {Broker};

    \node[draw,circle,fill=Firebrick3!20]
        (sub1) at ($(broker)+(5,0)$) {SUB};

    \draw
        (sub1)
        --
        ($(sub1.south)-(0,1.4)$)
        ;
    \draw
        (broker)
        --
        ($(broker.south)-(0,1.7)$)
        ;


    \draw[->]
        ($(sub1)-(0,.7)$)
        --
        node[midway,above]
        {\texttt{UNSUBSCRIBE}}
        ($(broker)-(0,.7)$)
        ;
    \draw[->]
        ($(broker)-(0,1.4)$)
        --
        node[midway,above] {\texttt{UNSUBACK}}
        ($(sub1)-(0,1.4)$)
        ;
\end{tikzpicture}

    \end{figure}
    El \texttt{UNSUBSCRIBE} indica uno/varios
    \texttt{topics}
    \begin{itemize}
        \item {\color{Firebrick3}Madrid}/{\color{Gold3}Comillas}/{\color{DodgerBlue3}Laura}/{\color{OliveDrab4}Presión}
        \item {\color{Firebrick3}Madrid}/{\color{Gold3}Comillas}/{\color{DodgerBlue3}+}/{\color{OliveDrab4}Pulso}
    \end{itemize}
\end{frame}





\begin{frame}{\secname}
    \textbf{Publicación} de información.
    \begin{figure}
        \begin{tikzpicture}
    \node[draw,circle,fill=DodgerBlue3!20] (pub1) at (0,0) {PUB};

    \node[draw,fill=yellow!20] (broker)
        at ($(pub1)+(5,0)$) {Broker};

    \draw
        (pub1)
        --
        ($(pub1.south)-(0,1.3)$)
        ;
    \draw
        (broker)
        --
        ($(broker.south)-(0,1.5)$)
        ;


    \draw[->]
        ($(pub1)-(0,1)$)
        --
        node[midway,above] {\texttt{PUBLISH}}
        ($(broker)-(0,1)$)
        ;
\end{tikzpicture}

    \end{figure}
    \emph{Pregunta}: ¿y si se pierde
    el mensaje \texttt{PUBLISH}?
\end{frame}



\section{QoS: Quality of Service}
\begin{frame}{\secname}
    MQTT permite tres niveles de calidad de servicio:
    \begin{itemize}
        \item \textbf{QoS 0}: puede haber pérdidas;
        \item \textbf{QoS 1}: llega el mensaje; y
        \item \textbf{QoS 2}: llega el mensaje sin duplicados.
    \end{itemize}
\end{frame}



\begin{frame}{\secname}
    \textbf{QoS 0}: no se reenvían mensajes perdidos.
    \begin{figure}
        \begin{tikzpicture}
    \node[draw,circle,fill=DodgerBlue3!20]
        (pub1) at (0,0) {PUB};

    \node[draw,fill=yellow!20] (broker)
        at ($(pub1)+(5,0)$) {Broker};

    \draw
        (pub1)
        --
        ($(pub1.south)-(0,1.3)$)
        ;
    \draw
        (broker)
        --
        ($(broker.south)-(0,1.5)$)
        ;


    \draw[->]
        ($(pub1)-(0,1)$)
        --
        node[midway,above] {\texttt{PUBLISH}}
        ($(broker)-(2,1)$)
        node[anchor=west,Firebrick3]
        {\textbf{X}}
        ;
\end{tikzpicture}

    \end{figure}
\end{frame}



\begin{frame}{\secname}
    \textbf{QoS 1}: se reenvía si no hay \texttt{PUBACK}.
    \begin{figure}
        \begin{tikzpicture}
    \node[draw,circle,fill=DodgerBlue3!20]
        (pub1) at (0,0) {PUB};

    \node[draw,fill=yellow!20] (broker)
        at ($(pub1)+(7,0)$) {Broker};

    \draw
        (pub1)
        --
        ($(pub1.south)-(0,3.3)$)
        ;
    \draw
        (broker)
        --
        ($(broker.south)-(0,3.5)$)
        ;


    \draw[->]
        ($(pub1)-(0,1)$)
        --
        node[midway,above] {\texttt{PUBLISH,Pkt ID=15,DUP=0}}
        ($(broker)-(2,1)$)
        node[anchor=west,Firebrick3]
        {\textbf{X}}
        ;

    \draw[->]
        ($(pub1)-(0,2)$)
        --
        node[midway,above] {\texttt{PUBLISH,Pkt ID=15,DUP=1}}
        ($(broker)-(0,2)$)
        ;
    \draw[->]
        ($(broker)-(0,3)$)
        --
        node[midway,above] {\texttt{PUBACK,Pkt ID=15}}
        ($(pub1)-(0,3)$)
        ;
\end{tikzpicture}

    \end{figure}
\end{frame}



\begin{frame}{\secname}
    \textbf{QoS 1}: si se pierde \texttt{PUBACK}, se retransmite.
    \begin{figure}
        \begin{tikzpicture}
    \node[draw,circle,fill=DodgerBlue3!20]
        (pub1) at (0,0) {PUB};

    \node[draw,fill=yellow!20] (broker)
        at ($(pub1)+(7,0)$) {Broker};

    \draw
        (pub1)
        --
        ($(pub1.south)-(0,4.3)$)
        ;
    \draw
        (broker)
        --
        ($(broker.south)-(0,4.5)$)
        ;


    \draw[->]
        ($(pub1)-(0,1)$)
        --
        node[midway,above] {\texttt{PUBLISH,Pkt ID=15,DUP=0}}
        ($(broker)-(0,1)$)
        node[draw,anchor=west,circle,fill=OliveDrab3!20,
        inner sep=1]
        {OK}
        ;

    \draw[->]
        ($(broker)-(0,2)$)
        --
        node[midway,above] {\texttt{PUBACK,Pkt ID=15}}
        ($(pub1)-(-2,2)$)
        node[anchor=east,Firebrick3]
        {\textbf{X}}
        ;

    \draw[->]
        ($(pub1)-(0,3)$)
        --
        node[midway,above] {\texttt{PUBLISH,Pkt ID=15,DUP=1}}
        ($(broker)-(0,3)$)
        node[draw,anchor=west,circle,fill=OliveDrab3!20,
        inner sep=1]
        {OK}
        ;


    \draw[->]
        ($(broker)-(0,4)$)
        --
        node[midway,above] {\texttt{PUBACK,Pkt ID=15}}
        ($(pub1)-(0,4)$)
        ;
\end{tikzpicture}

    \end{figure}
    El mensaje 15 se recibe y procesa \emph{dos veces}.
\end{frame}



\begin{frame}{\secname}
    \textbf{QoS 1}: procesa como nuevo cada \texttt{DUP=1}.
    \begin{figure}
        \begin{tikzpicture}
    \node[draw,circle,fill=DodgerBlue3!20]
        (pub1) at (0,0) {PUB};

    \node[draw,fill=yellow!20] (broker)
        at ($(pub1)+(7,0)$) {Broker};

    \draw
        (pub1)
        --
        ($(pub1.south)-(0,4.5)$)
        ;
    \draw
        (broker)
        --
        ($(broker.south)-(0,5)$)
        ;


    \draw[->]
        ($(pub1)-(0,1)$)
        --
        node[midway,above] {\texttt{PUBLISH,Pkt ID=15,DUP=0,`A'}}
        ($(broker)-(0,1)$)
        node[draw,anchor=west,circle,fill=OliveDrab3!20,
        inner sep=1]
        (ok1)
        {OK}
        ;

    \node[draw,anchor=west,fill=Firebrick4!20]
        at (ok1.east)
        {process};

    \draw[->]
        ($(broker)-(0,2)$)
        --
        node[midway,above] {\texttt{PUBACK,Pkt ID=15}}
        ($(pub1)-(0,2)$)
        ;

    \draw[->]
        ($(pub1)-(0,3)$)
        --
        node[midway,above] {\texttt{PUBLISH,Pkt ID=15,DUP=0,`B'}}
        ($(broker)-(1,3)$)
        node[anchor=west,Firebrick3]
        {\textbf{X}}
        ;

    \draw[->]
        ($(pub1)-(0,4)$)
        --
        node[midway,above] {\texttt{PUBLISH,Pkt ID=15,DUP=1,`B'}}
        ($(broker)-(0,4)$)
        node[draw,anchor=west,circle,fill=OliveDrab3!20,
        inner sep=1]
        (ok2)
        {OK}
        ;

    \node[draw,anchor=west,fill=Firebrick4!20]
        at (ok2.east)
        {process};

    \draw[->]
        ($(broker)-(0,5)$)
        --
        node[midway,above] {\texttt{PUBACK,Pkt ID=15}}
        ($(pub1)-(0,5)$)
        ;
\end{tikzpicture}

    \end{figure}
    El último PUBLISH es procesado como
    nuevo, si no se descarta pensando
    que es otro `A'.
\end{frame}



\begin{frame}{\secname}
    \textbf{QoS 1}: puede dar pie a mucho procesado duplicado.
    \begin{figure}
        \resizebox{.9\textwidth}{!}{
            \begin{tikzpicture}
    \node[draw,circle,fill=DodgerBlue3!20]
        (pub1) at (0,0) {PUB};

    \node[draw,fill=yellow!20] (broker)
        at ($(pub1)+(6,0)$) {Broker};

    \node[draw,circle,fill=Firebrick3!20]
        (sub1)
        at ($(pub1)+(11,0)$) {SUB};


    \draw
        (pub1)
        --
        ($(pub1.south)-(0,4.5)$)
        ;
    \draw
        (broker)
        --
        ($(broker.south)-(0,5)$)
        ;
    \draw
        (sub1)
        --
        ($(sub1.south)-(0,4.5)$)
        ;


    \draw[->]
        ($(pub1)-(0,1)$)
        --
        node[midway,above] {\small\texttt{PUBLISH,Pkt ID=15,DUP=0,`A'}}
        ($(broker)-(0,1)$)
        ;

    \draw[->]
        ($(broker)-(0,1.3)$)
        --
        node[midway,above] {\small\texttt{PUBLISH,Pkt ID=15,DUP=0,`A'}}
        ($(sub1)-(0,1.3)$)
        node[draw,anchor=west,circle,fill=OliveDrab3!20,
        inner sep=1]
        (ok1)
        {OK}
        ;
    \draw[->]
        ($(sub1)-(0,1.8)$)
        --
        node[midway,above] {\small\texttt{PUBACK,Pkt ID=15}}
        ($(broker)-(0,1.8)$)
        ;


    \draw[->]
        ($(broker)-(0,2)$)
        --
        node[midway,above] {\texttt{PUBACK,Pkt ID=15}}
        ($(pub1)-(-1,2)$)
        node[anchor=east,Firebrick3]
        {\textbf{X}}
        ;

    \draw[->]
        ($(pub1)-(0,3)$)
        --
        node[midway,above] {\small\texttt{PUBLISH,Pkt ID=15,DUP=1,`A'}}
        ($(broker)-(0,3)$)
        ;
    \draw[->]
        ($(broker)-(0,3.3)$)
        --
        node[midway,above] {\small\texttt{PUBLISH,Pkt ID=15,DUP=1,`A'}}
        ($(sub1)-(0,3.3)$)
        node[draw,anchor=west,circle,fill=OliveDrab3!20,
        inner sep=1]
        (ok2)
        {OK}
        ;
    \draw[->]
        ($(sub1)-(0,3.8)$)
        --
        node[midway,above] {\small\texttt{PUBACK,Pkt ID=15}}
        ($(broker)-(-1,3.8)$)
        node[anchor=east,Firebrick3]
        {\textbf{X}}
        ;

    \draw[->]
        ($(broker)-(0,4.5)$)
        --
        node[midway,above] {\small\texttt{PUBLISH,Pkt ID=15,DUP=1,`A'}}
        ($(sub1)-(0,4.5)$)
        node[draw,anchor=west,circle,fill=OliveDrab3!20,
        inner sep=1]
        (ok3)
        {OK}
        ;
    \draw[->]
        ($(sub1)-(0,5)$)
        --
        node[midway,above] {\small\texttt{PUBACK,Pkt ID=15}}
        ($(broker)-(0,5)$)
        ;



    \draw[->]
        ($(broker)-(0,4)$)
        --
        node[midway,above] {\texttt{PUBACK}}
        ($(pub1)-(0,4)$)
        ;


    \foreach \i in {1,2,3}
        \node[anchor=west,draw,fill=Firebrick4!20]
            at (ok\i.east)
            {process};

\end{tikzpicture}

        }
    \end{figure}
\end{frame}






\begin{frame}{\secname}
    \textbf{QoS 2}: se garantiza \emph{único} procesado.

    \begin{figure}
        \resizebox{.7\textwidth}{!}{
            \begin{tikzpicture}
    \node[draw,circle,fill=DodgerBlue3!20]
        (pub1) at (0,0) {PUB};

    \node[draw,fill=yellow!20] (broker)
        at ($(pub1)+(7,0)$) {Broker};

    \draw
        (pub1)
        --
        ($(pub1.south)-(0,4.5)$)
        ;
    \draw
        (broker)
        --
        ($(broker.south)-(0,5)$)
        ;


    \draw[->]
        ($(pub1)-(0,1)$)
        --
        node[midway,above] {\texttt{PUBLISH,Pkt ID=15,DUP=0,`A'}}
        ($(broker)-(0,1)$)
        ;


    \draw[->]
        ($(broker)-(0,2)$)
        --
        node[midway,above] {\texttt{PUBREC,Pkt ID=15}}
        ($(pub1)-(0,2)$)
        ;

    \draw[->]
        ($(pub1)-(0,3)$)
        --
        node[midway,above] {\texttt{PUBREL,Pkt ID=15}}
        ($(broker)-(0,3)$)
        ;

    \draw[->]
        ($(broker)-(0,4)$)
        --
        node[midway,above] {\texttt{PUBCOMP,Pkt ID=15}}
        ($(pub1)-(0,4)$)
        ;



    \path[draw,Firebrick3,ultra thick]
        (pub1.south west)
        --
        (broker.south east |- pub1.south west)
        --
        node[midway,anchor=west]
        {Pkt ID=15}
        ($(broker)-(-.3,4.3)$)
        --
        ($(pub1)-(.3,4.3)$)
        --
        cycle
        ;


    \draw[->]
        ($(pub1)-(0,5)$)
        --
        node[midway,above] {\texttt{PUBLISH,Pkt ID=16}}
        ($(broker)-(0,5)$)
        ;
\end{tikzpicture}

        }
    \end{figure}

    \begin{itemize}
        \item PUBREC: notifica la recepción;
        \item PUBREL: notifica liberar el Pkt ID; y
        \item PUBCOMP: notifica el fin de la transmisión.
    \end{itemize}
\end{frame}


\section{Pila de Protocolos}
\begin{frame}{\secname}
    MQTT (nivel aplicación) funciona sobre la pila TCP/IP:
    \begin{itemize}
        \item TCP: protocolo de transporte; y
        \item IP: protocolo de red (las direcciones).
    \end{itemize}
    \begin{figure}
        \usetikzlibrary{backgrounds,calc,positioning}
\begin{tikzpicture}
    \node[rectangle,draw,minimum width=10,fill=DodgerBlue3!20]
        (mqtt) {MQTT};
    \node[rectangle,draw,anchor=north,fill=yellow!20,
        minimum width={width("MQTT")+5pt}]
        (tcp) at (mqtt.south) {TCP};
    \node[rectangle,draw,anchor=north,fill=Firebrick3!20,
        minimum width={width("MQTT")+5pt}] (ip) at (tcp.south) {IP};
\end{tikzpicture}

    \end{figure}

    TCP garantiza envío mediante
    \texttt{tcp\_retries2}=15 reintentos.
\end{frame}




\section{Detalles Analíticos}
\begin{frame}{\secname}
    Suponga que se envía \texttt{PUBLISH}
    (QOS 0)
    cada $T$\,\textrm{sec}.
    Si $p$ es la probabilidad de éxito
    en transmisión, ¿cuál es el tiempo
    medio entre \texttt{PUBLISH}es
    recibidos con éxito?

    \begin{equation}
        \mathbb{E}[\text{IPG}] =
        \sum_{i=1}^\infty i T p(1-p)^{i-1}
        = \frac{Tp}{1-p}\sum_{i=1}^\infty i(1-p)^i
        = \frac{T}{p}
    \end{equation}

    \emph{Nota}: IPG apela al \textit{inter-packet gap}.
\end{frame}



\begin{frame}{\secname}
    Si la conexión está apagada un tiempo
    $t\sim exp(\lambda)$, ¿cuál es el
    número medio de mensajes \texttt{PUBLISH}
    perdidos $M$ con QoS 0?

    \begin{equation}
        \mathbb{E}[M]=\sum_{i=1}^\infty i \mathbb{P}(t>iT)
        =\sum_i^{\infty} i e^{-\lambda iT}
        =\frac{e^{-\lambda T}}{(e^{-\lambda T}-1)^2}
    \end{equation}


    \begin{figure}
\begin{tikzpicture}[scale=.5]
    \begin{axis}[ymode=log,xlabel={$\lambda$},ylabel={$\mathbb{E}[M]$},grid=major]
    \addplot[domain=0:10,line width=3,color=DodgerBlue4]
        {exp(-x*2)/(exp(-x*2)-1)^2};
\end{axis}
\end{tikzpicture}

\emph{Nota}: $\lambda$ es la inversa del tiempo sin conexión.
\end{figure}


\end{frame}





\section{Manos a la Obra con \texttt{Mosquitto}}
\begin{frame}{\secname}
    \texttt{Mosquitto} es un broker de MQTT. Provee de:
    \begin{itemize}
        \item clientes por línea de comando; y
        \item una librería en \texttt{C} para
            programar clientes.
    \end{itemize}
    Existen otras librerías como \texttt{Paho}
    para el lenguaje \texttt{Python}.
\end{frame}



\begin{frame}[fragile]{\secname}
    En primer lugar tenemos que instalar
    los paquetes en la máquina virtual:
\begin{lstlisting}[language=bash]
$ sudo apt install -y mosquitto
$ sudo apt install -y mosquitto-clients
\end{lstlisting}

    Ahora inicie el servicio 
    del broker y compruebe su estado:
\begin{lstlisting}[language=bash]
$ sudo service mosquitto start 
$ sudo service mosquitto status 
\end{lstlisting}
\end{frame}


\begin{frame}[fragile]{\secname}
    Ahora creamos un cliente que
    se \textbf{suscriba} a un \texttt{topic}:
\begin{lstlisting}[language=bash]
$ mosquitto_sub -V mqttv311 -t Madrid/Moncloa/__Nombre__ -d
\end{lstlisting}
sustituye \verb|__Nombre__| por el tuyo.


Inicie una captura \texttt{wireshark} antes
de ejecutar la línea. Busque:
\begin{enumerate}
    \item mensaje \texttt{CONNECT};
    \item mensaje \texttt{CONNACK};
    \item el ID del cliente;
    \item QoS indicada;
    \item los mensajes \texttt{SUBSCRIBE} \& \texttt{SUBACK}; y
    \item mensajes \texttt{Keepalive} (los \texttt{PING}s).
\end{enumerate}
\end{frame}

















\begin{frame}[fragile]{\secname}
    Ahora \textbf{publicamos} en el \texttt{topic}
    {\color{Firebrick3}Madrid}/{\color{Gold3}Moncloa}/{\color{DodgerBlue3}Nombre}/{\color{OliveDrab4}Pulso}:
\begin{lstlisting}[language=bash]
$ mosquitto_pub -V mqttv311\
   -t Madrid/Moncloa/__Nombre__/Pulso -m "60%" -d
\end{lstlisting}


sustituye \verb|__Nombre__| por el tuyo
y ejecuta en \emph{otra terminal}. Busque:
\begin{enumerate}
    \item el \texttt{PUBLISH};
    \item QoS indicada; y
    \item el campo \texttt{Message}.
\end{enumerate}

\vspace{1em}

\emph{Pregunta}: ¿por qué no sale 60 en \texttt{Message}?

\end{frame}






\begin{frame}[fragile]{\secname}
    Ahora creamos un cliente que publique
    con QoS 1:
\begin{lstlisting}[language=bash]
$ mosquitto_sub  -V mqttv311\
    -t Madrid/Moncloa/__Nombre__/Pulso -d\
    -q 1 -c -i 'client-qos1'
\end{lstlisting}
Vuelva a capturar con \texttt{wireshark} y busque
\begin{enumerate}
    \item el \texttt{PUBLISH}; y
    \item el \texttt{PUBACK}.
\end{enumerate}

Repita el proceso con QoS 2 y busque:
\begin{enumerate}
    \item el \texttt{PUBLISH};
    \item el \texttt{PUBREC}; 
    \item el \texttt{PUBREL}; y
    \item el \texttt{PUBCOMP}.
\end{enumerate}
\end{frame}





\end{document}




