\documentclass[xcolor={x11names}]{beamer}
\usetheme{Madrid}

\usepackage{amssymb}
\usepackage{ulem}
\usepackage[utf8]{inputenc}
\usepackage{mathtools}
\usepackage{multicol}
%\usepackage[x11names]{xcolor}
\usefonttheme{professionalfonts}


% Subfigures
\usepackage{caption}
\usepackage{subcaption}


% License
\usepackage[
    type={CC},
    modifier={by-nc-sa},
    version={4.0},
    imagewidth=4pt
]{doclicense}




% Change base colour beamer@blendedblue (originally RGB: 0.2,0.2,0.7)
% \colorlet{beamer@blendedblue}{DarkSeaGreen4}







%% MATH commands
\DeclareMathOperator{\Var}{Var}


%% THEOREMS
%\newtheorem{theorem}{Theorem}
\newtheorem{thm}{Teorema}[section] % the main one
% Definición
%\theoremstyle{definition}
\newtheorem{definicion}{Definición}[section]
\newtheorem{lema}{Lema}[section]



%% PFGplots %%
\usepackage{pgfplots}

%% Exponential distribution
\pgfmathdeclarefunction{exponential}{1}{%
  \pgfmathparse{(#1)*exp(-#1*x)}%
}
\pgfmathdeclarefunction{exponentialcdf}{1}{%
  \pgfmathparse{1-exp(-#1*x)}%
}

%% Poisson distribution
\pgfmathdeclarefunction{poiss}{1}{%
  \pgfmathparse{(#1^x)*exp(-#1)/(x!)}%
}

%% Normal distribution (#1=mu, #2=sigma)
% John D. Cook approx. https://tex.stackexchange.com/a/124629
\pgfmathdeclarefunction{normalcdf}{2}{%
  \pgfmathparse{1/(1 + exp(-0.07056*((x-#1)/#2)^3 - 1.5976*(x-#1)/#2))}%
}




\newcommand{\red}[1]{{\color{red}#1}}
\newcommand{\blue}[1]{{\color{blue}#1}}

%%%%%%%%%%
%% TIKZ %%
%%%%%%%%%%
\usepackage{tikz}
\usepackage{animate}
\usetikzlibrary{positioning}
\usetikzlibrary{shapes,arrows, positioning, calc}
\usetikzlibrary{overlay-beamer-styles}
\usetikzlibrary{chains,shapes.multipart}
\usetikzlibrary{scopes}
\usetikzlibrary{automata}
\usetikzlibrary{positioning}  %                 ...positioning nodes
\usetikzlibrary{arrows}       %                 ...customizing arrows
\usetikzlibrary{intersections}


%%%%%%%%%
%% PGF %%
%%%%%%%%%
\usepgfplotslibrary{fillbetween}


%%% Insert section name before the section %%%
\AtBeginSection[]{
  \begin{frame}
  \vfill
  \centering
  \begin{beamercolorbox}[sep=8pt,center,shadow=true,rounded=true]{title}
    \usebeamerfont{title}\insertsectionhead\par%
  \end{beamercolorbox}
  \vfill
  \end{frame}
}



\title[MQTT]{MQTT: fundamentos del protocolo}
%% \subtitle{Redes y Servicios de Telecomunicaciones (RSTC)\\
%% Grado en Ingeniería de Tecnologías y Servicios de Telecomunicación}
%\author{M. Saiful Bari\inst{1} \and Mr X\inst{2}}
\titlegraphic{%
\doclicenseIcon {\tiny \hspace{1em}\doclicenseText}
}

\author{\textcolor{white}{RSER curso 2024-2025}}
%\author{Jorge Martín Pérez\inst{1}}
%\institute{
%    \inst{1}
%    Departamento de Ingeniería Telemática, Universidad Politécnica de Madrid
%}

\date{\today}







%%%%%%%%%%%%%%%%%%%%
%%% SLIDES START %%%
%%%%%%%%%%%%%%%%%%%%
\begin{document}


%%% TITLE %%%
\frame{\titlepage }


\begin{frame}[allowframebreaks]{Contenido}
    \tableofcontents
\end{frame}




\section{Introducción}
\begin{frame}{\secname}
    MQTT (Message Queuing Telemetry Transport)
    se utiliza para:
    \begin{itemize}
        \item transportar mensajes (e.g. pulsaciones);
        \item utilizando esquema \textbf{publish/subscribe}.
    \end{itemize}

    \vfill
    Vamos a ver:
    \begin{itemize}
        \item en qué consiste el \textbf{publish/subscribe}; y
        \item cómo funciona el protocolo.
    \end{itemize}
\end{frame}




\begin{frame}{\secname}
    Entidades modelo publish/subscribe:
    \begin{itemize}
        \item \textbf{Publisher}: publica mensajes,
            e.g. 60\,\textrm{BPM};
        \item \textbf{Broker}: recibe, almacena, y envía mensajes;
        \item \textbf{Subscriber}: recibe mensajes.
    \end{itemize}
    \vspace{1em}
    \begin{figure}
        \input{figs/pub-subs}
    \end{figure}
\end{frame}



\begin{frame}{\secname}
    En qué casos se usa publish/subscribe:
    \begin{itemize}
        \item comunicaciones tolerantes a \textbf{latencias};
        \item escenarios con \textbf{muchos dispositivos}
            (escalabilidad).
    \end{itemize}
    \vspace{1em}
    \begin{figure}
        \begin{tikzpicture}
    \node[draw,circle,fill=blue!20] (pub1) at (0,0) {PUB};

    \node[draw,fill=yellow!20] (broker)
        at ($(pub1)+(3,0)$) {Broker};

    \node[draw,circle,fill=red!20] (subs)
        at ($(broker)+(3,0)$) {SUB};


    \draw[->] (pub1.east) --
        node[midway,above] {publish}
        (broker.west);
    \draw[->] (broker.east) --
        node[midway,above] {subscribe}
        (subs.west);
\end{tikzpicture}

    \end{figure}
\end{frame}




\section{Topics}
\begin{frame}{\secname}
    Los \texttt{topics} son como ``buzones''
    en los que se publican mensajes:
    \begin{itemize}
        \item tienen \textbf{jerarquía}.
    \end{itemize}

    \vspace{1em}

    \begin{minipage}{.5\textwidth} %
        {\color{Firebrick3}Madrid}/{\color{Gold3}Comillas}/{\color{DodgerBlue3}Laura}/{\color{OliveDrab4}Presión}
        {\color{Firebrick3}Madrid}/{\color{Gold3}Comillas}/{\color{DodgerBlue3}Laura}/{\color{OliveDrab4}Pulso}
        {\color{Firebrick3}Madrid}/{\color{Gold3}Comillas}/{\color{DodgerBlue3}Sofía}/{\color{OliveDrab4}Pulso}
        {\color{Firebrick3}Madrid}/{\color{Gold3}Alcorcón}/{\color{DodgerBlue3}Sofía}/{\color{OliveDrab4}Pulso}
    \end{minipage}
    \begin{minipage}{.45\textwidth} %
        \begin{figure}
            \input{figs/topics-tree}
        \end{figure}
    \end{minipage}
\end{frame}




\begin{frame}{\secname}
    \textbf{Single level wildcard (+)}:
    nos permite sustituir un solo nivel.

    \vspace{1em}

    
    \begin{minipage}{.51\textwidth} %
        {\color{Firebrick3}Madrid}/{\color{Gold3}Alcorcón}/{\color{DodgerBlue3}\textbf{+}}/{\color{OliveDrab4}Pulso} resulta en:

        \begin{itemize}
            \item {\color{Firebrick3}Madrid}/{\color{Gold3}Alcorcón}/{\color{DodgerBlue3}\textbf{Sofía}}/{\color{OliveDrab4}Pulso}
        \end{itemize}
    \end{minipage}
    \begin{minipage}{.4\textwidth} %
        \begin{figure}
            \begin{tikzpicture}
    \node[draw,fill=OliveDrab3!20]
        (prele) at (0,0) {Presión};
    \node[draw,fill=OliveDrab3!20,
        anchor=west]
        (pulsole) at
        ($(prele.east)+(.1,0)$)
        {Pulso};
    \node[draw,fill=OliveDrab3!20,
        anchor=west]
        (preri) at
        ($(pulsole.east)+(.1,0)$)
        {Presión};
    \node[draw,fill=OliveDrab3!20,
        anchor=west]
        (pulsori) at
        ($(preri.east)+(.1,0)$)
        {Pulso};

    \node[draw,fill=DodgerBlue3!20,
        anchor=south]
        (laura)
        at
        ($(prele)+(.5,.75)$)
        {Laura};
    \node[draw,fill=DodgerBlue3!20,
        anchor=south]
        (sofia)
        at
        ($(preri)+(.5,.75)$)
        {Sofía};



    \node[draw,fill=Gold2!20,
        anchor=south]
        (comillas)
        at
        ($(laura)+(.5,.75)$)
        {Comillas};
    \node[draw,fill=Gold2!20,
        anchor=west]
        (alcorcon)
        at
        ($(comillas.east)+(.5,0)$)
        {Alcorcón};

    \node[draw,fill=Firebrick3!20,
        anchor=south]
        (madrid)
        at
        ($(comillas)+(.5,.75)$)
        {Madrid};


    \draw (madrid.south) -- (comillas.north);
    \draw (madrid.south) -- (alcorcon.north);
    \draw (comillas.south) -- (laura.north);
    \draw (comillas.south) -- (sofia.north);
    \draw (alcorcon.south) -- (sofia.north);
    \draw (laura.south) -- (prele.north);
    \draw (laura.south) -- (pulsole.north);
    \draw (sofia.south) -- (preri.north);
    \draw (sofia.south) -- (pulsori.north);

        
    \draw[ultra thick] (madrid.south)
        to[out=-10,in=90]
        (alcorcon.north)
        to[out=-90,in=90]
        (pulsori.north);

\end{tikzpicture}

        \end{figure}
    \end{minipage}
\end{frame}








\begin{frame}{\secname}
    \textbf{Multi level wildcard (\#)}:
    nos permite sustituir muchos niveles.

    \vspace{1em}

    
    \begin{minipage}{.51\textwidth} %
        {\color{Firebrick3}Madrid}/{\color{Gold3}Alcorcón}/{\color{DodgerBlue3}\textbf{\#}} resulta en:

        \begin{itemize}
            \item {\color{Firebrick3}Madrid}/{\color{Gold3}Alcorcón}/{\color{DodgerBlue3}\textbf{Sofía}}/{\color{OliveDrab4}Pulso}
            \item {\color{Firebrick3}Madrid}/{\color{Gold3}Alcorcón}/{\color{DodgerBlue3}\textbf{Sofía}}/{\color{OliveDrab4}Presión}
        \end{itemize}
    \end{minipage}
    \begin{minipage}{.4\textwidth} %
        \begin{figure}
            \input{figs/topics-tree-multilevel-wildcard}
        \end{figure}
    \end{minipage}
\end{frame}






\section{PUBLISH}
\begin{frame}{\secname}
    El publisher envía información mediante
    mensajes de tipo \texttt{PUBLISH}.

    \vspace{1em}

    \begin{figure}
        \begin{tikzpicture}
    \node[draw,circle,fill=DodgerBlue3!20] (pub1) at (0,0) {PUB};

    \node[draw,fill=yellow!20] (broker)
        at ($(pub1)+(9,0)$) {Broker};


    \draw[->] (pub1.east) --
        node[midway,above,draw,align=center]
        (pubmsg)
        {PUBLISH\\
        Topic:{\color{Firebrick3}Madrid}/{\color{Gold3}Comillas}/{\color{DodgerBlue3}Laura}/{\color{OliveDrab4}Pulso}\\
        Message: 60\,\texttt{BMP}}
        (broker.west);

    \node[anchor=north] at
        ($(broker.south)-(0,.5)$)
    {
        \scalebox{.5}{
            \input{figs/topics-tree.tex}
        }
    };

    \draw
        (broker.south)
        to[out=-130,in=90]
        ($(broker)-(2,1)$)
        to[out=-90,in=-180]
        ($(broker)-(.7,3)$)
        node[anchor=west,draw,inner sep=1]
        (msgstore)
        {\tiny Message}
        ;

    \draw[->]
        (msgstore) -- ($(msgstore)+(0,.3)$);
\end{tikzpicture}

    \end{figure}
    
\end{frame}



\begin{frame}{\secname}
    El broker anuncia información
    a los suscriptores con
    mensajes \texttt{PUBLISH}.

    \vspace{1em}

    \begin{figure}
        \begin{tikzpicture}
    \node[draw,fill=yellow!20]
        (broker) at (0,0) {Broker};

    \node[draw,circle,fill=Firebrick3!20]
        (subs)
        at ($(broker)+(9,0)$) {SUB};



    \draw[->] (broker.east) --
        node[midway,above,draw,align=center]
        (pubmsg)
        {PUBLISH\\
        Topic:{\color{Firebrick3}Madrid}/{\color{Gold3}Comillas}/{\color{DodgerBlue3}Laura}/{\color{OliveDrab4}Pulso}\\
        Message: 60\,\texttt{BMP}}
        (subs.west);

    \node[anchor=north] at
        ($(broker.south)-(0,.5)$)
    {
        \scalebox{.5}{
            \input{figs/topics-tree.tex}
        }
    };

    \draw
        (broker.south)
        to[out=-130,in=90]
        ($(broker)-(2,1)$)
        to[out=-90,in=-180]
        ($(broker)-(.7,3)$)
        node[anchor=west,draw,inner sep=1]
        (msgstore)
        {\tiny Message}
        ;

    \draw[->]
        (msgstore) -- ($(msgstore)+(0,.3)$);
\end{tikzpicture}

    \end{figure}
\end{frame}



\section{SUBSCRIBE}
\begin{frame}{\secname}
    Un dispositivo se suscribe a
    un \texttt{topic} con
    mensajes \texttt{SUBSCRIBE}.

    \vspace{1em}

    \begin{figure}
        \input{figs/subscribe}
    \end{figure}


    \emph{Nota}: podríamos usar el wildcard
    {\color{Firebrick3}Madrid}/{\color{Gold3}Alcorcón}/{\color{DodgerBlue3}Laura}/{\color{OliveDrab4}\textbf{+}}
\end{frame}




\begin{frame}{\secname}
    En cuanto el broker recibe un
    mensaje nuevo, lo publica a los
    suscriptores.
    \begin{figure}
        \begin{tikzpicture}
    \foreach \i in {0,...,4}
        \node (pub\i) at (0,\i*.6) {};

    \node[draw,fill=yellow!20] (broker)
        at ($(pub2)+(3,0)$) {Broker};

    \foreach \i in {0,...,4}
        \node[draw,circle,fill=Firebrick3!20]
            (subs\i)
            at ($(pub\i)+(6,0)$) {SUB\i};


    \foreach \i in {0,...,4}{
        \draw[->] (broker.east) --
            node[above,midway] (suba\i) {}
            (subs\i.west);
    }

    \node[rotate=40] at (suba4.north)
        {\texttt{PUBLISH}};
\end{tikzpicture}

    \end{figure}
\end{frame}




\section{KEEPALIVE}
\begin{frame}{\secname}
    Los sensores sufren pérdidas de
    conexión. Las detectamos con
    \texttt{KEEPALIVE}.
    
    \begin{figure}
        \begin{tikzpicture}
    \node[draw,fill=yellow!20]
        (broker) at (0,0) {Broker};

    \node[draw,circle,fill=Firebrick3!20]
        (subs)
        at ($(broker)+(9,0)$) {SUB};



    \draw[->] (subs.west) --
        node[midway,above,draw,align=center]
        (pubmsg)
        {SUBSCRIBE\\
        Topics:\\
        {\color{Firebrick3}Madrid}/{\color{Gold3}Comillas}/{\color{DodgerBlue3}Laura}/{\color{OliveDrab4}Pulso}\\
        {\color{Firebrick3}Madrid}/{\color{Gold3}Comillas}/{\color{DodgerBlue3}Laura}/{\color{OliveDrab4}Presión}}
        (broker.east);



\end{tikzpicture}

    \end{figure}
\end{frame}




\end{document}




