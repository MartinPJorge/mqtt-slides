\documentclass[xcolor={x11names}]{beamer}
\usetheme{Madrid}

\usepackage{amssymb}
\usepackage{ulem}
\usepackage[utf8]{inputenc}
\usepackage{mathtools}
\usepackage{multicol}
%\usepackage[x11names]{xcolor}
\usefonttheme{professionalfonts}


% Subfigures
\usepackage{caption}
\usepackage{subcaption}


% License
\usepackage[
    type={CC},
    modifier={by-nc-sa},
    version={4.0},
    imagewidth=4pt
]{doclicense}




% Change base colour beamer@blendedblue (originally RGB: 0.2,0.2,0.7)
% \colorlet{beamer@blendedblue}{DarkSeaGreen4}







%% MATH commands
\DeclareMathOperator{\Var}{Var}


%% THEOREMS
%\newtheorem{theorem}{Theorem}
\newtheorem{thm}{Teorema}[section] % the main one
% Definición
%\theoremstyle{definition}
\newtheorem{definicion}{Definición}[section]
\newtheorem{lema}{Lema}[section]



%% PFGplots %%
\usepackage{pgfplots}

%% Exponential distribution
\pgfmathdeclarefunction{exponential}{1}{%
  \pgfmathparse{(#1)*exp(-#1*x)}%
}
\pgfmathdeclarefunction{exponentialcdf}{1}{%
  \pgfmathparse{1-exp(-#1*x)}%
}

%% Poisson distribution
\pgfmathdeclarefunction{poiss}{1}{%
  \pgfmathparse{(#1^x)*exp(-#1)/(x!)}%
}

%% Normal distribution (#1=mu, #2=sigma)
% John D. Cook approx. https://tex.stackexchange.com/a/124629
\pgfmathdeclarefunction{normalcdf}{2}{%
  \pgfmathparse{1/(1 + exp(-0.07056*((x-#1)/#2)^3 - 1.5976*(x-#1)/#2))}%
}




\newcommand{\red}[1]{{\color{red}#1}}
\newcommand{\blue}[1]{{\color{blue}#1}}

%%%%%%%%%%
%% TIKZ %%
%%%%%%%%%%
\usepackage{tikz}
\usepackage{animate}
\usetikzlibrary{positioning}
\usetikzlibrary{shapes,arrows, positioning, calc}
\usetikzlibrary{overlay-beamer-styles}
\usetikzlibrary{chains,shapes.multipart}
\usetikzlibrary{scopes}
\usetikzlibrary{automata}
\usetikzlibrary{positioning}  %                 ...positioning nodes
\usetikzlibrary{arrows}       %                 ...customizing arrows
\usetikzlibrary{intersections}


%%%%%%%%%
%% PGF %%
%%%%%%%%%
\usepgfplotslibrary{fillbetween}


%%% Insert section name before the section %%%
\AtBeginSection[]{
  \begin{frame}
  \vfill
  \centering
  \begin{beamercolorbox}[sep=8pt,center,shadow=true,rounded=true]{title}
    \usebeamerfont{title}\insertsectionhead\par%
  \end{beamercolorbox}
  \vfill
  \end{frame}
}



\title[MQTT]{MQTT: fundamentos del protocolo}
%% \subtitle{Redes y Servicios de Telecomunicaciones (RSTC)\\
%% Grado en Ingeniería de Tecnologías y Servicios de Telecomunicación}
%\author{M. Saiful Bari\inst{1} \and Mr X\inst{2}}
\titlegraphic{%
\doclicenseIcon {\tiny \hspace{1em}\doclicenseText}
}

\author{\textcolor{white}{RSER curso 2024-2025}}
%\author{Jorge Martín Pérez\inst{1}}
%\institute{
%    \inst{1}
%    Departamento de Ingeniería Telemática, Universidad Politécnica de Madrid
%}

\date{\today}







%%%%%%%%%%%%%%%%%%%%
%%% SLIDES START %%%
%%%%%%%%%%%%%%%%%%%%
\begin{document}


%%% TITLE %%%
\frame{\titlepage }


\begin{frame}[allowframebreaks]{Contenido}
    \tableofcontents
\end{frame}




\section{Introducción}
\begin{frame}{\secname}
    MQTT (Message Queuing Telemetry Transport)
    se utiliza para:
    \begin{itemize}
        \item transportar mensajes (e.g. pulsaciones);
        \item utilizando esquema \textbf{publish/subscribe}.
    \end{itemize}

    \vfill
    Vamos a ver:
    \begin{itemize}
        \item en qué consiste el \textbf{publish/subscribe}; y
        \item cómo funciona el protocolo.
    \end{itemize}
\end{frame}




\begin{frame}{\secname}
    Entidades modelo publish/subscribe:
    \begin{itemize}
        \item \textbf{Publisher}: publica mensajes,
            e.g. 60\,\textrm{BPM};
        \item \textbf{Broker}: recibe, almacena, y envía mensajes;
        \item \textbf{Subscriber}: recibe mensajes.
    \end{itemize}
    \vspace{1em}
    \begin{figure}
        \begin{tikzpicture}
    \node[draw,circle,fill=blue!20] (pub1) at (0,0) {PUB};

    \node[draw,fill=yellow!20] (broker)
        at ($(pub1)+(3,0)$) {Broker};

    \node[draw,circle,fill=red!20] (subs)
        at ($(broker)+(3,0)$) {SUB};


    \draw[->] (pub1.east) --
        node[midway,above] {publish}
        (broker.west);
    \draw[->] (broker.east) --
        node[midway,above] {subscribe}
        (subs.west);
\end{tikzpicture}

    \end{figure}
\end{frame}



\begin{frame}{\secname}
    En qué casos se usa publish/subscribe:
    \begin{itemize}
        \item comunicaciones tolerantes a \textbf{latencias};
        \item escenarios con \textbf{muchos dispositivos}
            (escalabilidad).
    \end{itemize}
    \vspace{1em}
    \begin{figure}
        \input{figs/pub-subs-many}
    \end{figure}
\end{frame}



\end{document}

